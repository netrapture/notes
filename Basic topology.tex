\documentclass[11pt]{amsart}
%\documentclass[oneside]{book}
\usepackage{geometry} % see geometry.pdf on how to lay out the page. There's lots.
\geometry{a4paper} % or letter or a5paper or ... etc
% \geometry{landscape} % rotated page geometry
\usepackage{enumitem}
\usepackage{amsthm}

\usepackage{etoolbox}
\AtBeginEnvironment{example}{\tiny}
\AtBeginEnvironment{remark}{\small}
\AtBeginEnvironment{verify}{\tiny}

% See the ``Article customise'' template for come common customisations

\title{Basic topology}
\author{}
\date{} % delete this line to display the current date

%%% BEGIN DOCUMENT
\begin{document}

\maketitle{}
\thispagestyle{empty} 
%\tableofcontents

%\section{}
%\subsection{}

\theoremstyle{definition}
\newtheorem*{definition}{Definition}
\newtheorem*{verify}{Verify}
\newtheorem*{remark}{Remark}
\newtheorem*{example}{Example}
\newtheorem*{lemma}{Lemma}
\newtheorem*{note}{Note}
\newtheorem*{theorem}{Theorem}

\newcommand{\st}{\ni :}


\section{Set theory preliminaries}
%\subsection{}

\begin{note}{(\textbf{Empty families})}
$\newline$ 
Let A be an empty index set and  let $(A_{\alpha})_{\alpha \in A}$ be a family of subsets of X indexed by A.
$\newline$ 
$\bigcup\limits_{\alpha \in A} A_{\alpha} = \{x \in X | \exists \; \alpha \in A, x \in A_{\alpha} \}$ and since there is no $\alpha \st x \in A_{\alpha}$ so $\bigcup\limits_{\alpha \in A} A_{\alpha} = \emptyset$.
$\newline$ 
$\bigcap\limits_{\alpha \in A} A_{\alpha} = \{x \in X | \forall \alpha \in A, x \in A_{\alpha} \} =  \emptyset$, because $\not \exists \alpha \st x \in  A_{\alpha}$. 
$\newline$ 
Recall $\prod\limits_{\alpha \in A} A_{\alpha} = \{f: A \rightarrow \bigcup\limits_{\alpha \in A} A_{\alpha}\}$. When $A = \emptyset$ then  $\bigcup\limits_{\alpha \in A} A_{\alpha} = \emptyset$ and no function has an empty range so the set $\prod\limits_{\alpha \in A} A_{\alpha} = \emptyset$
\end{note}

\begin{note}{(\textbf{Subsets of product set})}
$U_1 \times V_1 \cap U_2 \cap V_2 = (U_1 \cap U_2) \times (V_1 \times V_2)$
\begin{verify} () 
\begin{enumerate}
\item (Show $U_1 \times V_1 \cap U_2 \cap V_2 \subseteq (U_1 \cap U_2) \times (V_1 \times V_2)$)
If $(x,y) \in U_1 \times V_1 \cap U_2 \cap V_2 $ then $(x,y) \in U_1 \times V_1$ so $x \in U_1$ and $y \in V_1$.
Also $(x,y) \in  U_2 \cap V_2$ so $x \in U_2$ and $y \in V_2$. So $x \in U_1$ and $x \in U_2$ so $x \in U_1 \cap U_2$. Similarly, $y \in V_1 \cap V_2$. So $(x,y) \in (U_1 \cap U_2) \times (V_1 \cap V_2)$ 
\item (Show $U_1 \times V_1 \cap U_2 \cap V_2 \supseteq (U_1 \cap U_2) \times (V_1 \times V_2)$).
If $(x,y) \in (U_1 \cap U_2) \times (V_1 \times V_2)$ then $x \in U_1 \cap U_2$ and $y \in (V_1 \cap V_2)$. Then $x \in U_1$ and $y \in V_1$ so $(x,y) \in U_1 \times V_1$. Similarly,  $(x,y) \in U_2 \times V_2$. So $(x,y) \in U_1 \times V_1 \cap  U_2 \times V_2$
\end{enumerate}
\end{verify}
\end{note}

\begin{note}{(\textbf{Properties of bijections})}
\begin{enumerate}
\item If f is a bijection, $f^{-1}$ is a bijection
\item if f is a bijection, $((f^{-1})^{-1} = f$
\end{enumerate}
\end{note}

\section{Topological spaces}
\subsection{Base for a topology}

\begin{definition}{(\textbf{Base for a topology})} A collection $\mathcal{B}$ of subsets of a set X is $\textbf{a base for a topology}$ (on X) if it satisfies:
\begin{enumerate}[label=(\arabic*)]
\item $\mathcal{B}$ is a cover:  $\forall x \in X \;  \exists B \in \mathcal{B} \st x \in B$
\item If $B_1, B_2 \in \mathcal{B}$ and $x \in B_1 \cap B_2$ then $\exists \; B_3 \in \mathcal{B} \st x \in B_3 \subseteq B_1 \cap B_2$
\end{enumerate}
\end{definition}

\begin{remark}
Property 2 of a base can be  extended from the  intersection of two base elements to any finite intersection viz. whenever $x \in \bigcap\limits_{i =1}^n B_i \;  \exists B \in \mathcal{B} \st x \in B \subseteq  \bigcap\limits_{i  =1}^n B_i$ (P(n))
\begin{verify} (Proof by induction) 
\begin{enumerate}
\item P(1):  $(B_i) = \{B_1\}$. Say $x \in \bigcap\limits_{i  =1}^1 B_i = B_1\;  \exists B \in \mathcal{B} \st x \in B \subseteq B_1$ namely $B=B_1$
\item (IH) Assume P(k): $x \in \bigcap\limits_{i  =1}^k \implies  \exists B \in \mathcal{B} \st x \in B \subseteq  \bigcap\limits_{i  =1}^k B_i$
\item Consider P(k+1):  Say $x \in \bigcap\limits_{i  =1}^{k+1} B_i = (\bigcap\limits_{i  =1}^k B_i) \cap B_{k+1}$ so $x \in \bigcap\limits_{i  =1}^k B_i$ and $x \in B_{k+1}$. By (IH) $x \in \bigcap\limits_{i  =1}^k B_i \implies \exists B' \in \mathcal{B} \st x \in B' \subseteq \bigcap\limits_{i  =1}^k B_i$ and by the property of a base  $x \in B' \cap B_{k+1} \implies \exists B'' \in \mathcal{B} \st x \in B'' \subseteq B' \cap B_{k+1} \subseteq \bigcap\limits_{i  =1}^k \cap B_{k+1} = \bigcap\limits_{i  =1}^{k+1} B_i$
\item Thus the proposition holds for all $n \in \mathcal{N}: x \in \bigcap\limits_{i =1}^n B_i \implies \;  \exists B \in \mathcal{B} \st x \in B \subseteq  \bigcap\limits_{i  =1}^n B_i$
\end{enumerate}
\end{verify}
\end{remark}



\begin{example}{\textbf{(Example of a base)}} $\mathcal{B}_{std}  = \{B_r(p) | p \in \mathbb{R}^2, r \in \mathbb{R}^+\}$ is a base for a topology.
\begin{verify}  { (  $\mathcal{B}_{std} \textbf{ is a base}$  )       }
\begin{enumerate}
\item $\forall (x,y) =: p \in \mathbb{R}^2$,  $p \in B_r(p)$ for any $r \in \mathbb{R}^+$, so every point is contained in a base element, so  $\mathcal{B}_{\mathbb{R}^2}$ is a cover.
\item If $p \in B_{r_1}(p_1) \cap  B_{r_2}(p_2)$  then $p \in B_r(p) \subseteq B_{r_1}(p_1) \cap  B_{r_2}(p_2)$ for $r = \min_{i=1,2} \{r_i - d(p,r_i)\} -\epsilon, \; \forall \epsilon > 0$
\end{enumerate}
\end{verify}
\end{example}


\begin{example}
%$\newline$
$\mathcal{B}  = \{\textnormal{interiors of all rectangles in } \mathbb{R}^2$ is a base for a topology.
\begin{verify} { (  $\mathcal{B}$ is a base)       }
%$\newline$ 
\begin{enumerate}
\item $\mathcal{B}$  is a cover of $\mathbb{R}^2$ since $\forall p=(x,y) \in \mathbb{R}^2, p \in (x-a,x+a) \times (y-b,y+b)$  for any $a,b > 0$, so every point is contained in a base element.
\item If $p \in B_1 \cap  B_2$, then $p \in B_3 \subseteq B_1 \cap B_2$ where
\end{enumerate}
\end{verify}
\end{example}




\begin{definition}{[Topology generated by a base]}
If $\mathcal{B}$ is a base for a topology on X, then the $\textbf{topology generated}$ by it is defined by:
U is open in $\mathcal{T}_{\mathcal{B}}$ if $\forall x \in U, \exists B \in \mathcal{B} \ni: x \in B \subseteq U$
\end{definition}



\begin{verify}{[the topology generated by a base is a topology]}
$\newline$ 
Let the collection $\mathcal{B} \subseteq \mathcal{P}(X)$ be a base for a topology, $\mathcal{T}_{\mathcal{B}}$
\begin{enumerate}
\item $\emptyset$ and X $\in \mathcal{T}_{\mathcal{B}}$ follows from vacuity and being a cover
\item Let $(U_i)_{i=1}^n$ be open sets in $\mathcal{T}_{\mathcal{B}}$. 
Say  $x \in \bigcap\limits_{i=1}^n U_i$. 
Show $\exists B \in \mathcal{T}_{\mathcal{B}} \st x \in B \subseteq  \bigcap\limits_{i=1}^n U_i$. 
$x \in \bigcap\limits_{i=1}^n U_i \implies \forall i \; x \in U_i$. 
Since $\forall i \; U_i$ is open in $\mathcal{T}_{\mathcal{B}}$ then $ \forall i \;  \exists B_i \in \mathcal{T}_{\mathcal{B}} \st x \in B_i \subseteq U_i \implies x \in \bigcap\limits_{i=1}^n B_i \subseteq \bigcap\limits_{i=1}^n U_i$. And by the previous result $x \in \bigcap\limits_{i=1}^n B_i \implies \exists B \in \mathcal{T}_{\mathcal{B}}$ with $x \in B \subseteq \bigcap\limits_{i=1}^n B_i \subseteq \bigcap\limits_{i=1}^n U_i$, so $\bigcap\limits_{i=1}^n U_i$ is open in $\mathcal{T}_{\mathcal{B}}$
\item Let $(U_{\alpha})_{\alpha \in A}$ be open sets in $\mathcal{T}_{\mathcal{B}}$. Show $\forall x \in \bigcup\limits_{\alpha \in A} U_{\alpha} \; \exists B \in \mathcal{T}_{\mathcal{B}} \st x \in B \subseteq \bigcup\limits_{\alpha \in A} U_{\alpha}$. $x \in \bigcup\limits_{\alpha \in A} U_{\alpha}  \implies \exists \tilde{\alpha} \in A \st x \in U_{ \tilde{\alpha}}$. Then since $U_{ \tilde{\alpha}}$ is open$ \exists B \in \mathcal{T}_{\mathcal{B}} \st x \in B \subseteq U_{ \tilde{\alpha}}$. But $B \subseteq U_{ \tilde{\alpha}} \implies B \subseteq \bigcup\limits_{\alpha \in A} U_{\alpha}$ so $\bigcup\limits_{\alpha \in A} U_{\alpha}$ is shown to be open.
\end{enumerate}
\end{verify}



\begin{remark}
The elements, B, of a base are open sets in the topology generated by the base since $\forall x \in B, x \in B \subseteq B$
\end{remark}



\begin{example}
Example of topology defined by a base
$\newline$ 
Let $(X,\mathcal{T}_X)$ and $(Y,\mathcal{T}_Y)$ be topological spaces.
Consider the set $X \times Y$. Then $\mathcal{B} = \{ U \times V|$  U open in X and V open in Y\} is a base for a topology.
\begin{verify} 
$\newline$ 
\end{verify}
\end{example}

\begin{lemma}
For a set X, if $\mathcal{B}$ is a base for a topology $\mathcal{T}$ on X, then  $\mathcal{T}$ is the collection of all unions of all elements of $\mathcal{B}$ .
\begin{proof} Show $\mathcal{T}$= the collection of all unions of all elements of $\mathcal{B}$
\begin{enumerate}
\item ($\subseteq$) Say $U \in \mathcal{T}$. Then U is open, so $\forall x \in U \; \exists B_x \in \mathcal{B}$ with $x \in B_x \subseteq U$. 
Claim $U = \bigcup\limits_{x \in X} B_x$
\begin{enumerate}
\item ($U \subseteq  \bigcup\limits_{x \in X} B_x$) Say $x \in U$. Then $x \in B_x \subseteq \bigcup\limits_{x \in X} B_x$
\item ($ \bigcup\limits_{x \in X} B_x \subseteq U$) Say $\tilde{x} \in \bigcup\limits_{x \in X} B_x$. Then $\exists x' \st \tilde{x} \in B_{x'}$ But $B_{x'} \subseteq U$ so $\tilde{x} \in U$
\end{enumerate}
\item  ($\supseteq$) Say B is a union of elements of  $\mathcal{B}$. Show B is open wrt the base. So $B = \bigcup\limits_{\alpha \in A} B_{\alpha}$ and say $x \in B$. Then $x \in B_{\alpha_0}$ for some $\alpha_0 \in A$ and $B_{\alpha_0} \subseteq B$. So B is open
\end{enumerate}
\end{proof}
\end{lemma}

\begin{lemma}
For a space X, suppose $\mathcal{C}$ is a collection of open sets of X such that for each open set U of X and each $x \in U$, there is an element C of   $\mathcal{C}$ such that $x \in C \subseteq U$. Then $\mathcal{C}$ is a base for the topology on X.
\end{lemma}
\begin{proof}
Need to show $\mathcal{C}$ is a base and the topology it generates is the given topology on X
\begin{enumerate}[label=(\alph*)]
\item ($\mathcal{C}$ is a base)
\begin{enumerate}
\item ($\mathcal{C}$ covers X)
X itself is an open set of X, so $\forall x \in X \;  \exists C \in  \mathcal{C} \st x \in C ( \subseteq X)$ so  $\mathcal{C}$ is a cover.
\item ($C_1, C_2 \in \mathcal{C}$ and $x \in C_1 \cap C_2$ then $\exists \; C_3 \in \mathcal{C} \st x \in C_3 \subseteq C_1 \cap C_2$) Say $x \in C_1 \cap C_2$ with $C_1, C_2 \in  \mathcal{C}$ Since $\mathcal{C}$  is a collection of open sets, $C_1 \cap C_2$ is open and therefore if $x \in C_1 \cap C_2 \exists C_3 \in \mathcal{C} \st x \in C_3 \subseteq C_1 \cap C_2$
\end{enumerate}
\item (the topology it generates is the given topology, viz. U is open in $\mathcal{T}_{\mathcal{C}}$ i.e. $\forall x \in U \;  \exists B \in  \mathcal{C}$ with $x \in B \subseteq U$
\end{enumerate}
\end{proof}

\subsection{Subbase for a topology}

\begin{definition}{Subbase for a topology}
$\newline$
A subbase $\mathcal{S}$ for a topology on X is cover of X and $\textbf{the topology generated by the subbase}$ is the collection $\mathcal{T}$ of unions of finite intersections of elements of  $\mathcal{S}$
\end{definition}
\begin{verify} (the topology generated by a subbase is indeed a topology)
$\newline$ 
\begin{enumerate}
\item 
\item 
\item
\end{enumerate}
\end{verify}

\subsection{Subspace topology}

\begin{definition}{Subspace topology}
$\newline$ 
For a set $Y \subseteq (X,\mathcal{T}_X)$, $\mathcal{T}_Y = \{U \cap Y | U \in \mathcal{T}_X\}$ is a topology for Y, called $\textbf{the subspace topology}$ on Y.
\end{definition}

\begin{verify} (The subspace topology $\mathcal{T}_Y = \{U \cap Y | U \in \mathcal{T}_X\}$ is a topology for Y)
\begin{enumerate}
\item $\emptyset = \emptyset \cap Y$ and $Y = X \cap Y$
\item If $(V_i)_{1 \le i \le k} $ are open sets in Y, then $\exists (U_i)_{1 \le i \le k}$ open in U $\st \bigcap\limits_{i=1}^k V_i = \bigcap\limits_{i=1}^k (U_i \cap Y) = (\bigcap\limits_{i=1}^k U_i) \cap Y$ and $\bigcap\limits_{i=1}^k U_i$ is open in X so  $\bigcap\limits_{i=1}^k U_i) \cap Y$ is open in Y.
\item If $(V_{\alpha})_{\alpha \in A} $ are open sets of Y, then $\exists (U_{\alpha})_{\alpha \in A} $ open in sets in X $\st V_{\alpha} = U_{\alpha} \cap Y$. Then $\bigcup V_{\alpha} = \bigcup  (U_{\alpha} \cap Y) = (\bigcup  U_{\alpha}) \cap Y$ and $\bigcup  U_{\alpha}$ is open in X so $(\bigcup  U_{\alpha}) \cap Y$ is open in Y
\end{enumerate}
\end{verify}

\begin{lemma}
$\newline$
If $Y \subseteq (X,\mathcal{T}_X)$ and $\mathcal{B}$ is a base for the topology on X, then $\{B \cap Y| B \in \mathcal{B}\} =: \mathcal{B}_Y$ is a base for the subspace topology on Y.
\begin{verify}
$\newline$
\begin{enumerate}
\item (cover) If $x \in Y$, then $x \in X$ so since $\mathcal{B}$ is a base for X, it is a cover for X so $\exists B \in \mathcal{B} \st x \in B$ so $x \in B \cap Y$ and $ B \cap Y \in  \mathcal{B}_Y$ so it is a cover
\item If $x \in Y$ and $x \in B_1 \cap B_2$ then $x \in B'_1 \cap Y  \cap B'_2 \cap Y$ for base elements $B'_1,B''_2$ of X. So $x \in  B'_1 \cap B'_2$ so $\exists B'_3 \st x \in  B'_1 \cap B'_2 \subseteq B'_3$. Then $B_3 = B'_3 \cap Y$ satisfies $x \in  B_1 \cap B_2 \subseteq B_3$
\end{enumerate}
\end{verify}
\end{lemma}

\subsection{Product topology}
\begin{definition}{(\textbf{Product topology})}
If X and Y are topological spaces then the collection $\mathcal{B}_{X \times Y}$ of subsets of $X \times Y$ of the form $U \times V$ where U is open in X and V is open in Y is a base for a topology. 
\end{definition}
\begin{verify}
$\newline$
\begin{enumerate}
\item (cover)  $X \times Y$  is covered by the single product subset $X \times Y$
\item (intersections) If $(x,y) \in U_1 \times V_1 \cap U_2 \times V_2$
\end{enumerate}
\end{verify}

\section{Continuous maps}

\begin{definition}{(\textbf{Continuous map})}
 Let $(X,\mathcal{T}_X)$ and $(Y,\mathcal{T}_Y)$ be topological spaces.
 A map $f X: \rightarrow Y$ is called $\textbf{continuous}$ if for all V open in Y, $preim_f(V)$ is open in X.
\end{definition}

\begin{remark}
If the topology on Y has a base, then continuity of a map to Y can be defined in terms of the base elements of Y, since
 if V is open in Y
 then V is a union of base elements, $V = \bigcup B_{\alpha}$
 and the preimage set map respects unions: $preim_f(V) = preim_f(\bigcup B_{\alpha}) =\bigcup preim_f(B_{\alpha})$, so if all the $preim_f(B_{\alpha})$ are open then $preim_f(V)$ is open since it can expressed as a union of  preimages of base elements.
\end{remark}

\begin{remark}
If the topology on Y is given by a subbase $\mathcal{S}$, then to show continuity of a map, it suffices to show that the preimage under the map of the subbase elements are open since
 if V is open in Y
 then V is a union of base elements, $V = \bigcup B_{\alpha}$ and each base element is a finite intersection of $n_{\alpha}$ subbase elements (indexed by $i_{\alpha}$)
 and the preimage set map respects intersections and unions: $preim_f(V) = preim_f(\bigcup\limits_{\alpha \in A} B_{\alpha}) = preim_f( \bigcup\limits_{\alpha \in A} \;\bigcap\limits_{        i_{\alpha} \in \{1, \dots, n_{\alpha} \} }  S_{i_{\alpha}})$  since $B_{\alpha} = \bigcap\limits_{i_{\alpha} \in \{1, \dots, n_{\alpha} \} } S_{i_{\alpha}}$
 \\
 Then $preim_f(V) = preim_f( \bigcup\limits_{\alpha \in A} \;\bigcap\limits_{        i_{\alpha} \in \{1, \dots, n_{\alpha} \} }  S_{i_{\alpha}}) = \bigcup\limits_{\alpha \in A} preim_f(\bigcap\limits_{        i_{\alpha} \in \{1, \dots, n_{\alpha} \} }  S_{i_{\alpha}} )= \bigcup\limits_{\alpha \in A}\; \bigcap\limits_{        i_{\alpha} \in \{1, \dots, n_{\alpha} \} }  preim_f(S_{i_{\alpha}})$, so if each of those preimages of subbase elements is open under f then the union of finite intersections of them is open.
 
 %\bigcup preim_f(B_{\alpha})$, so if all the $preim_f(B_{\alpha})$ are open then $preim_f(V)$ is open since it can expressed as a union of  preimages of base elememts.
\end{remark}

\begin{definition}{$\textbf{Continuous map at a point}$}
A mapping f of X to Y is $\textbf{continuous at } x_0 \in Y$ if for any nbhd V' of $f(x_0)$ in Y,  there is a nbhd V of $x_0$ in X such that the relation $x \in V \implies f(x) \in V'$
\end{definition}


\begin{theorem} $f:\mathbb{R} \rightarrow \mathbb{R}$ is $(\epsilon \text{--} \delta$)-continuous if and only if it is open-set continuous.
\end{theorem}
Recall $f:\mathbb{R} \rightarrow \mathbb{R}$ is $(\epsilon \text{--} \delta$)-continuous, if $\forall \epsilon > 0 \; \exists \delta > 0$ such that $|x - a| < \delta \implies |f(x) -f(a)| < \epsilon$
\begin{proof} Show iff
$\newline$ 
\begin{enumerate}[label=(\alph*)]
\item (If) If f is open-set continuous, then f is $(\epsilon \text{--} \delta$)-continuous
$\newline$ 
Note:$|f(x) -f(a)| < \epsilon$ is the interval $f(a) - \epsilon < f(x) < f(a) - \epsilon $
\item (Only if) If f is $(\epsilon \text{--} \delta$)-continuous, then f is open-set continuous.
\end{enumerate}
\end{proof}

%$\pagebreak$

\begin{definition}{Open/closed map} \\
 Let $(X,\mathcal{T}_X)$ and $(Y,\mathcal{T}_Y)$ be topological spaces. 
 A map $f X: \rightarrow Y$ is called $\textbf{open/closed}$ if for all A open/closed in X,  f(A) is open/closed in Y. The canonical projection maps $pr_i$
\end{definition}

%\theoremstyle{definition}
$\newline$ 
%\newtheorem*{exmp}{Example}
\begin{example}
Consider the space $X \times Y$ with the product topology $ \mathcal{T}_{X \times Y}$
\end{example}

$\newline$ 
$\newline$ 
\begin{definition}{($\textbf{Homeomorphism}$)} A map $f:X \rightarrow Y$ is a $\textbf{homeomorphism}$ if f is bijective, f is continuous and $f^{-1}$, also a bijection,  is continuous
\end{definition}

\begin{theorem}{($\textbf{A homeomorphism induces a bijection on the topologies}$)}
\end{theorem}
\begin{proof}
Say  $f:X \rightarrow Y$ is a homeomorphism.
Define $\tilde{f}:\mathcal{T}_X \xrightarrow{\sim} \mathcal{T}_Y$ by $\tilde{f}(U) = f(U)$ and $\tilde{f}^{-1}: \mathcal{T}_Y \rightarrow \mathcal{T}_X$ by $\tilde{f}^{-1}(V) = f^{-1}(V)$
\begin{enumerate}
\item $\tilde{f}$ is well-defined requires $\tilde{f}$ is defined on its domain. If U is open $\tilde{f}(U) = f(U)$, and the map $im_{f}$ is defined on subsets of X. We must also show $\tilde{f}$ has  images in the specified codomain, but its images are images of f which ....$\tilde{f}(U) \in \mathcal{T}_Y$. Say $U \in  \mathcal{T}_X$.
$\tilde{f}(U) = f(U) = {(f^{-1}})^{-1}(U)$ and since $f^{-1}$ is continuous ${(f^{-1}})^{-1}(U) \in \mathcal{T}_Y$ as was to be shown.
\item 
\end{enumerate}
\end{proof}


\end{document}










